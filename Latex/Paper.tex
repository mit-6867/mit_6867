%%%%%%%%%%%%%%%%%%%%%%%%%%%%%%%%%%%%%%%%%
% Stylish Article
% LaTeX Template
% Version 2.1 (1/10/15)
%
% This template has been downloaded from:
% http://www.LaTeXTemplates.com
%
% Original author:
% Mathias Legrand (legrand.mathias@gmail.com) 
% With extensive modifications by:
% Vel (vel@latextemplates.com)
%
% License:
% CC BY-NC-SA 3.0 (http://creativecommons.org/licenses/by-nc-sa/3.0/)
%
%%%%%%%%%%%%%%%%%%%%%%%%%%%%%%%%%%%%%%%%%

%----------------------------------------------------------------------------------------
%	PACKAGES AND OTHER DOCUMENT CONFIGURATIONS
%----------------------------------------------------------------------------------------

\documentclass[fleqn,12pt]{SelfArx} % Document font size and equations flushed left

\usepackage[english]{babel} % Specify a different language here - english by default

\usepackage{lipsum} % Required to insert dummy text. To be removed otherwise

%----------------------------------------------------------------------------------------
%	COLUMNS
%----------------------------------------------------------------------------------------

\setlength{\columnsep}{0.55cm} % Distance between the two columns of text
\setlength{\fboxrule}{0.75pt} % Width of the border around the abstract

%----------------------------------------------------------------------------------------
%	COLORS
%----------------------------------------------------------------------------------------

\definecolor{color1}{RGB}{0,0,90} % Color of the article title and sections
\definecolor{color2}{RGB}{0,20,20} % Color of the boxes behind the abstract and headings

%----------------------------------------------------------------------------------------
%	HYPERLINKS
%----------------------------------------------------------------------------------------

\usepackage{hyperref} % Required for hyperlinks
\hypersetup{hidelinks,colorlinks,breaklinks=true,urlcolor=color2,citecolor=color1,linkcolor=color1,bookmarksopen=false,pdftitle={Title},pdfauthor={Author}}

%----------------------------------------------------------------------------------------
%	ARTICLE INFORMATION
%----------------------------------------------------------------------------------------

\JournalInfo{6.867 Final Paper} % Journal information
\Archive{} % Additional notes (e.g. copyright, DOI, review/research article)

\PaperTitle{How Does Content Drive Viewership?} % Article title

\Authors{Dave Holtz\textsuperscript{1}, Jeremy Yang\textsuperscript{2}, Michael Zhao\textsuperscript{3}} % Authors
\affiliation{\textsuperscript{1}\textit{dholtz@mit.edu}}
\affiliation{\textsuperscript{2}\textit{zheny@mit.edu}} 
\affiliation{\textsuperscript{3}\textit{mfzhao@mit.edu}}

\Keywords{} % Keywords - if you don't want any simply remove all the text between the curly brackets
\newcommand{\keywordname}{Keywords} % Defines the keywords heading name

%----------------------------------------------------------------------------------------
%	ABSTRACT
%----------------------------------------------------------------------------------------

\Abstract{Why do some webpages receive massive numbers of pageviews? To determine how content drives viewership, we construct a unique dataset of all articles published by the New York Times (NYT) in August 2013. Our dataset is built from 2 major components, the NYT's internal web traffic data and article content data parsed from the NYT website. We use the internal web traffic data to accurately track the number of page views of each article as well as construct a set of robust control variables such as the desk and section of each article. To build content features, we use various machine learning and statistical natural language processing techniques on our parsed article content data and construct features such as article perplexity, sentiment, reading difficulty, and indicators that denote the presence of pictures, videos, etc. Additionally, we have access to the NYT's internal website traffic data. We feed all of our constructed features to into a predictive regression model. We find [MAJOR RESULTS HERE].
}

%----------------------------------------------------------------------------------------

\begin{document}

\flushbottom % Makes all text pages the same height

\maketitle % Print the title and abstract box

\tableofcontents % Print the contents section

\thispagestyle{empty} % Removes page numbering from the first page

%----------------------------------------------------------------------------------------
%	ARTICLE CONTENTS
%----------------------------------------------------------------------------------------

\section{Introduction} % The \section*{} command stops section numbering

In today's digital economy, many companies are very interested in attracting users to visit their websites in order to earn ad revenue. While many factors might motivate a user to visit a particular page, certainly one important factor is the content in that webpage. This paper explores the relationship between the content of a webpage and the number of page views it ultimately ends up receiving by constructing a unique dataset of all articles published by the New York Times (NYT) during August 2013. This dataset is built from two major components: the NYT's internal web traffic data and parsed NYT article content data.

Typically, a study such as ours tends to be very difficult to conduct as either accurate measures of viewership are unavailable\footnote{While oftentimes precise viewership data tends to be not available openly, oftentimes researchers use related observables, such as Facebook likes} or the feature extraction of the content is too challenging (for example Youtube), or or both. Fortunately, our access to the the NYT's internal web traffic data allows us to exactly measure the number of page views an article receives. The web traffic data is rather rich and also includes internal meta-data that we use to build various control features. Moreover, since we are working with mostly textual data, we are able to take advantage of recent advancements in machine learning and statistical NLP to do feature extraction on article text. 

A similar study by Berger and Milkman (2012) \cite{BergerMilkman12} examines the relationship between content and word-of-mouth virality. They find that the emotional content of a NYT article is predictive of its virality. Using simple measures of an article's sentiment and emotionality, Berger and Milkman show that positive articles are more likely to show up on the New York Times "Most-Emailed" list. They also show that articles that evoke high physiological positive or negative arousal (such as awe or anger) tend to be more viral than articles that evoke deactivating emotions (sadness). We build on this study in two ways: first, we relate an article's content back to the number of page views it receives rather than its virality\footnote{Which companies arguably care more about since word-of-mouth virality is usually a means to increase page views}. Second, we employ more sophisticated machine learning feature extraction techniques to see if they work any better over their simple measures.
%------------------------------------------------

\section{Data}
\subsection{NYT Internal Web Traffic Data}
Our NYT internal web traffic dataset is a record of all individual user activity on the NYT website covering the period of April 3rd, 2013 to October 31st, 2013. This activity data is stored as individual lines of json and includes who (if available) accessed what page at what time. Overall, it is over 20 terabytes in size and contains over 3 billion page views\footnote{Not all page views are article views, for example, some events that are also tracked are searches, or user account settings.} Since the scope of this dataset is so large, we initially restrict this project to a single month, August 2013. 

Looking at only pages that only contain articles or blogposts, 

\subsection{Parsed NYT Article Content Data}

%\begin{figure*}[ht]\centering % Using \begin{figure*} makes the figure take up the entire width of the page
%\includegraphics[width=\linewidth]{view}
%\caption{Wide Picture}
%\label{fig:view}
%\end{figure*}


\begin{figure}[ht]\centering
\includegraphics[width=\linewidth]{results}
\caption{In-text Picture}
\label{fig:results}
\end{figure}

Reference to Figure \ref{fig:results}.

%------------------------------------------------

\section{}



\subsection{Subsection}


\begin{table}[hbt]
\caption{Table of Grades}
\centering
\begin{tabular}{llr}
\toprule
\multicolumn{2}{c}{Name} \\
\cmidrule(r){1-2}
First name & Last Name & Grade \\
\midrule
John & Doe & $7.5$ \\
Richard & Miles & $2$ \\
\bottomrule
\end{tabular}
\label{tab:label}
\end{table}



\begin{description}
\item[Word] Definition
\item[Concept] Explanation
\item[Idea] Text
\end{description}

\subsubsection{Subsubsection}


\begin{itemize}[noitemsep] % [noitemsep] removes whitespace between the items for a compact look
\item First item in a list
\item Second item in a list
\item Third item in a list
\end{itemize}

\subsubsection{Subsubsection}

\subsection{Subsection}



%------------------------------------------------
\phantomsection
\section*{Acknowledgments} % The \section*{} command stops section numbering

\addcontentsline{toc}{section}{Acknowledgments} % Adds this section to the table of contents

So long and thanks for all the fish \cite{Figueredo:2009dg}.

%----------------------------------------------------------------------------------------
%	REFERENCE LIST
%----------------------------------------------------------------------------------------
\phantomsection
\bibliographystyle{unsrt}
\bibliography{sample}

%----------------------------------------------------------------------------------------

\end{document}