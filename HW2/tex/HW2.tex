\documentclass[10pt]{article}
\usepackage[margin=1in]{geometry} 
\usepackage{enumerate, xfrac, color, graphicx}
\usepackage{amsmath,amsthm,amssymb,amsfonts,mathabx}
\usepackage{booktabs}
\usepackage{caption}
\usepackage{algorithm}
\usepackage{algpseudocode}
\usepackage{pifont}
\usepackage{listings, courier}
\graphicspath{{/Users/mfzhao/Dropbox/}}
\newcommand{\N}{\mathbb{N}}
\newcommand{\Z}{\mathbb{Z}}
\lstset{breaklines=true, basicstyle=\small\ttfamily, language=R, backgroundcolor=\color{highlight}, stepnumber=5}

\definecolor{highlight}{RGB}{248,248,248}

\begin{document}
	\title{6.867 Problem Set 2}
	\maketitle
	
\subsubsection*{Support Vector Machines}

Let's now compare the performance of SVM to that of logistic regression for classification problems. To illustrate the objective and constraints of the support vector machine, we have included below the explicit objective we would optimize over, as well as the constraints, for the dual form of a linear SVM with slack variables. The equations below correspond to the 2D problem where we have positive examples (1, 2), (2, 2) and negative examples (0, 0), (-2, 3).

\[
\min_{\alpha_1, \alpha_2, \alpha_3, \alpha_4}
\frac{1}{2}
\begin{bmatrix}
    \alpha_1 & \alpha_2 & \alpha_3 & \alpha_4 \\
\end{bmatrix}
\begin{bmatrix}
    5       & 6 & 0 & -4 \\
    6       & 8 & 0 & -2 \\
    0       & 0 & 0 & 0 \\
    -4       & -2 & 0 & 13 \\
\end{bmatrix}
\begin{bmatrix}
    \alpha_1 \\
    \alpha_2 \\
    \alpha_3 \\
    \alpha_4 \\
\end{bmatrix} 
+
\begin{bmatrix}
    -1       & -1 & -1 & -1 \\
\end{bmatrix}
\begin{bmatrix}
    \alpha_1 \\
    \alpha_2 \\
    \alpha_3 \\
    \alpha_4 \\
\end{bmatrix} 
\]

\begin{center}
s.t.
\end{center}

\[
\begin{bmatrix}
    -1 & 0 & 0 & 0 \\
    0 & -1 & 0 & 0 \\
    0 & 0 & -1 & 0 \\
    0 & 0 & 0 & -1 \\
    1 & 0 & 0 &0 \\
    0 & 1 & 0 & 0 \\
    0 & 0 & 1 & 0 \\
    0 & 0 & 0 & 1 \\
\end{bmatrix}
\begin{bmatrix}
    \alpha_1 \\
    \alpha_2 \\
    \alpha_3 \\
    \alpha_4 \\
\end{bmatrix} 
\leq
\begin{bmatrix}
0 \\
0 \\ 
0 \\ 
0 \\ 
C \\
C \\ 
C \\ 
C \\
\end{bmatrix},
\]

\[
\begin{bmatrix}
    1 & 1 & -1 & -1 \\
\end{bmatrix}
\begin{bmatrix}
    \alpha_1 \\
    \alpha_2 \\
    \alpha_3 \\
    \alpha_4 \\
\end{bmatrix} 
= 0
\]

\begin{table}
\captionof{table}{Performance of Linear SVM on provided data sets}
\begin{tabular}{llllll}
\toprule
{} & dataset & classification error rate (training set) & classification error rate (validation set) \\
\midrule
  & data\_stdev1 & 0.00\% & 0.00\% \\
  & data\_stdev2 & 9.50\% & 7.50\% \\
  & data\_stdev4 & 26.00\% & 23.50\% \\
  & data\_nonsep & 49.50\% & 51.25\% \\
\bottomrule
\end{tabular}
\end{table}

\begin{figure}[ht]
	\centering
	\begin{minipage}[b]{.24\linewidth}
		\includegraphics[width=1\linewidth, height=1in]{linear_svm_stdev1_train.png}
		\caption*{stdev1 (Training)}
	\end{minipage}
	\begin{minipage}[b]{.24\linewidth}
		\includegraphics[width=1\linewidth, height=1in]{linear_svm_stdev2_train.png}
		\caption*{stdev2 (Training)}
	\end{minipage}
	\begin{minipage}[b]{.24\linewidth}
		\includegraphics[width=1\linewidth, height=1in]{linear_svm_stdev4_train.png}
		\caption*{stdev4 (Training)}
	\end{minipage}
	\begin{minipage}[b]{.24\linewidth}
		\includegraphics[width=1\linewidth, height=1in]{linear_svm_nonsep_train.png}
		\caption*{nonsep (Training)}
	\end{minipage}
		\begin{minipage}[b]{.24\linewidth}
		\includegraphics[width=1\linewidth, height=1in]{linear_svm_stdev1_validation.png}
		\caption*{stdev1 (Validation)}
	\end{minipage}
	\begin{minipage}[b]{.24\linewidth}
		\includegraphics[width=1\linewidth, height=1in]{linear_svm_stdev2_validation.png}
		\caption*{stdev2 (Validation)}
	\end{minipage}
	\begin{minipage}[b]{.24\linewidth}
		\includegraphics[width=1\linewidth, height=1in]{linear_svm_stdev4_validation.png}
		\caption*{stdev4 (Validation)}
	\end{minipage}
	\begin{minipage}[b]{.24\linewidth}
		\includegraphics[width=1\linewidth, height=1in]{linear_svm_nonsep_validation.png}
		\caption*{nonsep (Validation)}
	\end{minipage}
	\caption{The decision boundaries generated by SVM plotted against the stdev1, stdev2, stdev4, and nonsep (training and validation) datasets}
\end{figure}


\end{document}
Status API Training Shop Blog About Pricing
© 2015 GitHub, Inc. Terms Privacy Security Contact Help